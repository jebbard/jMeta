%-----------------------------------------------------------------------------------------------
%		\COMPutility{} Design
%-----------------------------------------------------------------------------------------------

\section{\COMPutility{} Design}
\label{sec:COMPutilityDesign}

In diesem Abschnitt wird das Design der Komponente \COMPutility{} beschrieben. Grundaufgabe der Komponente ist das Anbieten genereller Querschnittsfunktionalit�t, die unabh�ngig von der Fachlichkeit ist, und so potentiell in mehreren Projekten Verwendung finden kann. Hier werden allerdings nur diejenigen Aspekte beschrieben, die f�r \LibName{} relevant sind.


%-----------------------------------------------------------------------------------------------
%		Configuration API
%-----------------------------------------------------------------------------------------------

\subsection{Configuration API}
\label{sec:ConfigurationAPI}

Die Configuration API bietet allgemeine Funktionen f�r Laufzeit-Konfiguration von Software-Komponenten an. Wir entwickeln hier das Design der API.

%%%% DD --> %%%%
\DD{dd:211}
{% Titel
Ein Konfigurationsparameter wird durch eine generische Klasse \ConfigProp{} repr�sentiert 
}
{% Kurzbeschreibung
Die Klasse \ConfigProp{} repr�sentiert einen konkreten Konfigurationsparameter, nicht jedoch dessen Wert an sich. Der Typparameter T gibt die Klasse des Wertes der Property an, dabei gilt: \texttt{T extends Comparable}. Instanzen der Klasse \ConfigProp{} werden als Konstanten in konfigurierbaren Klassen definiert.
Die Klasse hat folgende Eigenschaften und Funktionen:
\begin{itemize}
\item \texttt{getName()}: Der Name des Konfigurationsparameters
\item \texttt{getDefaultValue()}: Der Default-Wert des Konfigurationsparameters
\item \texttt{getMaximumValue()}: Den Maximal-Wert des Konfigurationsparameters oder null, falls er keinen hat
\item \texttt{getMinimumValue()}: Den Minimal-Wert des Konfigurationsparameters oder null, falls er keinen hat
\item \texttt{getPossibleValues()}: Eine Auflistung der m�glichen Werte, oder null falls er keine Auflistung fester Werte hat
\item \texttt{stringToValue()}: Konvertiert eine String-Repr�sentation eines Konfigurationsparameterwertes in den eigentlichen Datentyp des Wertes
\item \texttt{valueToString()}: Konvertiert den Wert eines Konfigurationsparameterwertes in eine Stringrepr�sentation
\end{itemize}
}
{% Begr�ndung
Eine solche Repr�sentation garantiert Typ-Sicherheit und eine bequeme Verwendung der API. Das Einschr�nken auf \texttt{Comparable} ist keine wirkliche Einschr�nkung, da so gut wie alle Wert-Klassen aus Java-SE, u.a. die numerischen Typen, Strings, Boolean, Character, Charset, Date, Calendar, diverse Buffer-Implementierungen usw. \texttt{Comparable} implementieren.
}
{% Nachteile
Keine bekannten Nachteile
}
%%%% <-- DD %%%%

Jede konfigurierbare Klasse soll m�glichst eine einheitliche Schnittstelle bereitstellen, um Konfigurationsparameter zu setzen und abzufragen:

%%%% DD --> %%%%
\DD{dd:212}
{% Titel
Schnittstelle f�r das Handhaben von Konfigurationsparametern
}
{% Kurzbeschreibung
Jede konfigurierbare Klasse muss die Schnittstelle \IConfigurable{} implementieren, mit folgenden Methoden:
\begin{itemize}
\item \texttt{setConfigParam()}: Setzt den Wert eines Konfigurationsparameters
\item \texttt{getConfigParam()}: Liefert den aktuellen Wert eines Konfigurationsparameters
\item \texttt{getAllConfigParams()}: Liefert alle Konfigurationsparameter mit ihren aktuellen Werten
\item \texttt{getSupportedConfigParams()}: Liefert ein Set aller von dieser Klasse unterst�tzten Konfigurationsparameter
\item \texttt{getAllConfigParamsAsProperties()}: Liefert alle Konfigurationsparameter als eine \texttt{Properties}-Instanz.
\item \texttt{configureFromProperties()}: Setzt die Werte aller Konfigurationsparameter basierend auf einer \texttt{Properties}-Instanz
\item \texttt{resetConfigToDefault()}: Setzt die Werte aller Konfigurationsparameter auf ihre Default-Werte zur�ck
\end{itemize}

Dabei m�ssen alle unterst�tzten Konfigurationsparameter der Klasse unterschiedliche Namen haben.
}
{% Begr�ndung
Damit wird eine einheitliche Schnittstelle f�r jede konfigurierbare Klasse erm�glicht. Die unterschiedlichen Namen sind zur eindeutigen Identifikation notwendig.
}
{% Nachteile
Keine bekannten Nachteile
}
%%%% <-- DD %%%%

Damit nun nicht jede Klasse selbst die Handhabung und Verifikation der Konfiguration implementieren muss, definieren wir:
%%%% DD --> %%%%
\DD{dd:213}
{% Titel
\ConfigurationHandler{} implementiert \IConfigurable{} und kann von jeder konfigurierbaren Klasse verwendet werden
}
{% Kurzbeschreibung
Die nicht-abstrakte Klasse \ConfigurationHandler{} implementiert \IConfigurable{} und �bernimmt die gesamte Aufgabe der Konfiguration. Sie kann von konfigurierbaren Klassen entweder als Basisklasse oder aber als aggregierte Instanz verwendet werden, an die alle Aufrufe weitergeleitet werden.
}
{% Begr�ndung
Keine Klasse muss die Konfigurationsverwaltung selbst implementieren
}
{% Nachteile
Keine bekannten Nachteile
}
%%%% <-- DD %%%%

Der Umgang mit fehlerhaften Konfiguration ist Teil der folgenden Designentscheidung:
%%%% DD --> %%%%
\DD{dd:214}
{% Titel
Fehlerhafte Konfigurationsparameterwerte f�hren zu einem Laufzeitfehler
}
{% Kurzbeschreibung
Wird ein fehlerhafter Wert f�r einen Konfigurationsparameter �bergeben, reagiert die API mit einem Laufzeitfehler, einer \texttt{InvalidConfigParamException}. Hierf�r wird eine �ffentliche Methode \texttt{checkValue()} in \ConfigProp{} bereitgestellt.
}
{% Begr�ndung
Die Wertebereiche der Parameter sind wohldefiniert und beschrieben, es handelt sich um einen Programmierfehler, wenn ein falscher Wert �bergeben wird.
}
{% Nachteile
Keine bekannten Nachteile
}
%%%% <-- DD %%%%

�nderungen von Konfigurationsparametern m�ssen u.U. sofort wirksam werden, daher definieren wir:

%%%% DD --> %%%%
\DD{dd:215}
{% Titel
Observer-Mechanismus f�r Konfigurations�nderungen
}
{% Kurzbeschreibung
Es wird ein Observer-Mechanismus �ber \IConfigurationChangeListener{} bereitgestellt, sodass Klassen �ber dynamische Konfigurations�nderungen informiert werden. Dieses Interface hat lediglich eine Methode \texttt{configurationParameterValueChanged()}.

\IConfigurable{} erh�lt damit zwei weitere Methoden: \texttt{registerConfigurationChangeListener()} und \texttt{unregisterConfigurationChangeListener()}.
}
{% Begr�ndung
Die konfigurierbaren Klassen m�ssen nicht immer diejenigen Klassen sein, welche mit den Konfigurations�nderungen umgehen m�ssen und die Konfigurationsparameter direkt verwenden. Stattdessen kann es sich um ein kompliziertes Klassengeflecht handeln, das zur Laufzeit keine direkte Beziehung hat. Daher ist ein entkoppelnder Listener-Mechanismus n�tig.
}
{% Nachteile
Keine bekannten Nachteile
}
%%%% <-- DD %%%%



%###############################################################################################
%###############################################################################################
%
%		File end
%
%###############################################################################################
%###############################################################################################