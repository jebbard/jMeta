%===============================================================================================
%		Minipages for tables heavily containing text
%===============================================================================================

% �berschriften-Minipage f�r Tag-Vergleichstabelle
\newenvironment{TAGcapt}{\begin{minipage}[t]{0.09\textheight}\begin{flushleft}}{\\\emph{}\end{flushleft}\end{minipage}}
\newenvironment{TAGcrit}{\begin{minipage}[t]{0.11\textheight}\begin{flushleft}}{\\\emph{}\end{flushleft}\end{minipage}}

% �berschriften-Minipage f�r ID3v2-Tag-Vergleichstabelle
\newenvironment{IDTAGcapt}{\begin{minipage}[t]{0.24\textheight}\begin{flushleft}}{\\\emph{}\end{flushleft}\end{minipage}}
\newenvironment{IDTAGcrit}{\begin{minipage}[t]{0.18\textheight}\begin{flushleft}}{\\\emph{}\end{flushleft}\end{minipage}}

% Beschreibungsminipage f�r Use-Case-Eigenschafts-�berschrift (in einer Use-Case-Tabelle)
\newenvironment{UCcapt}{\begin{minipage}[t]{0.26\textwidth}\begin{flushleft}}{\\\emph{}\end{flushleft}\end{minipage}}
% Minipage f�r Use-Case-Eigenschaften (in einer Use-Case-Tabelle)
\newenvironment{UCdesc}{\begin{minipage}[t]{0.74\textwidth}\begin{flushleft}}{\\\emph{}\end{flushleft}\end{minipage}}

% Beschreibungsminipage f�r Standard-Tabellen (zweispaltig)
\newenvironment{TABLEcapt}{\begin{minipage}[t]{0.32\textwidth}\begin{flushleft}}{\\\emph{}\end{flushleft}\end{minipage}}
% Minipage f�r Standard-Tabellen (zweispaltig)
\newenvironment{TABLEdesc}{\begin{minipage}[t]{0.68\textwidth}\begin{flushleft}}{\\\emph{}\end{flushleft}\end{minipage}}

% Minipage f�r Package-Eigenschaften (in einer Package-Beschreibungs-Tabelle)
\newenvironment{Packdesc}{\begin{minipage}[t]{0.36\textwidth}\begin{flushleft}}{\\\emph{}\end{flushleft}\end{minipage}}

%===============================================================================================
%		New Environments
%===============================================================================================

% Guideline-Environment
\newenvironment{Guideline}[1]{\textbf{#1}: }{}

%===============================================================================================
%		Code Listing Environments
%===============================================================================================

% Color of XML comments
\definecolor{XMLCommentColor}{rgb}{0.0,0.60,0.0}
% Color of XML tags
\definecolor{XMLTagColor}{rgb}{0.00,0.00,0.70}
% Color of XML attribute content
\definecolor{XMLAttributeContentColor}{rgb}{0.80,0.00,0.00}

% XML Style definition. It is no float figure, but with a caption. This way, it can be arbitrary long and is correctly wrapped to multiple pages.
\lstdefinestyle{XMLStyle}{language=XML,
% print the listing
print=true,		
% tabs are printed as consisting of 3 spaces
tabsize=3,		
% Whitespace characters are not shown
showspaces=false, showtabs=false,	
% Empty lines at end of listing are not shown
showlines=false,
% Show line numbers
numbers=left, numberstyle=\tiny, stepnumber=1, numbersep=10pt, firstline=1, numberblanklines=true,
% No spaces in strings
showstringspaces=false,
% XML tags are printed as keywords + their style
usekeywordsintag=true, tagstyle=\color{XMLTagColor},
% Smaller size
basicstyle=\footnotesize,
% Strings (=attribute contents) is printed in red
stringstyle=\color{XMLAttributeContentColor}\textbf, 
% XML comment, must stand at end of option list to get effective
morecomment=[s][\color{XMLCommentColor}]{<!--}{-->},
% Break lines that are too long and indent boken lines
breaklines=true, breakautoindent=true, breakindent=20pt
}

% Other Style definition. It is no float figure, but with a caption. This way, it can be arbitrary long and is correctly wrapped to multiple pages.
\lstdefinestyle{OtherStyle}{
% print the listing
print=true,		
% tabs are printed as consisting of 3 spaces
tabsize=3,		
% Whitespace characters are not shown
showspaces=false, showtabs=false,	
% Empty lines at end of listing are not shown
showlines=false,
% Smaller size
basicstyle=\footnotesize,
% Show line numbers
numbers=left, numberstyle=\tiny, stepnumber=1, numbersep=10pt, firstline=1, numberblanklines=true,
% Break lines that are too long and indent boken lines
breaklines=true, breakautoindent=true, breakindent=20pt
}

%===============================================================================================
%		Naming labels for referencing
%===============================================================================================

% Ensures that labels can be given arbitrary names which then
% can be referred to
% NOTES:
%  (1) requires \usepackage{namref} to work
%  (2) use \labelname right before \label, e.g. \labelname{XYZ 5000}\label{sec:XYZ}
%  (3) then you can e.g. use \nameref{sec:XYZ} which will display "XYZ 5000"
\makeatletter
\newcommand{\labelname}[1]{% \labelname{<stuff>}
  \def\@currentlabelname{#1}}%
\makeatother


%===============================================================================================
%		Open Issue Command
%===============================================================================================

% Counter f�r offene Punkte, der pro Kapitel hochz�hlt
\newcounter{OpenIssueCounter}[chapter]

% Hintergrundfarbe f�r Boxen von offenen Punkten
\definecolor{OpenIssueColor}{rgb}{0.97,0.97,0.71}

% Einf�gen offener Punkte in den Textfluss
\newcommand{\OpenIssue}%
[2]%
{\addtocounter{OpenIssueCounter}{1}%
	\vskip\baselineskip%
	\fcolorbox{black}{OpenIssueColor}%
	{%
		\begin{minipage}[t]{1\textwidth}%
			\begin{flushleft}%
				\emph{\textbf{Open Issue \the\value{chapter}.\the\value{OpenIssueCounter}:}} \--- \textbf{#1} \\%
			#2%
		\end{flushleft}%
	\end{minipage}%
}%
\index%
{%
	Open Issue \the\value{chapter}.\the\value{OpenIssueCounter}: \--- #1%
}%
\vskip\baselineskip%
}

%===============================================================================================
%		Design decision command
%===============================================================================================

% Counter for design decisions
\newcounter{DesignDecisionCounter}

% Hintergrundfarbe f�r Boxen von Designentscheidungen
\definecolor{DesignDecisionColor}{rgb}{0.85,0.95,0.95}

% Einf�gen von Designentscheidungen in den Textfluss
\newcommand{\DES}{DES~\ifnum\value{DesignDecisionCounter}<100 0\fi\ifnum\value{DesignDecisionCounter}<10 0\fi\arabic{DesignDecisionCounter}}

% #1 - The unique id of the design decision, to be used as a label, do not change after creation
% #2 - The title of the design decision
% #3 - The short description of the design decision
% #4 - The rationale of the design decision
% #5 - The disadvantages of the design decision
\newcommand{\DD}[5]{
\addtocounter{DesignDecisionCounter}{1}%
\labelname{\DES{}}\phantomsection\label{#1}
\vskip\baselineskip%
\fcolorbox{black}{DesignDecisionColor}%
{%
\begin{minipage}[t]{1\textwidth}%
   \begin{flushleft}%
     \emph{\textbf{\DES{}:}} \textbf{#2} \\%
     #3%
   \end{flushleft}%
   \begin{flushleft}%
     \textbf{Begr�ndung:} #4
   \end{flushleft}%
   \begin{flushleft}%
     \textbf{Nachteile:} #5
   \end{flushleft}%
\end{minipage}%
}%
\vskip\baselineskip%
}

\newcommand{\DesLink}[1]{\hyperref[#1]{\nameref{#1}}}

%===============================================================================================
%		(Hyper)link Command
%===============================================================================================

\newcommand{\SectionLink}[1]%
	{\hyperref[#1]{``\ref{#1}~\nameref{#1}''}}

%===============================================================================================
%		Basic Terms Command
%===============================================================================================

\newcommand{\BasicTermLink}[2]%
	{\emph{\hyperref[#2]{$\Rightarrow$\nolinebreak[4] #1}}}

%===============================================================================================
%		Special Terms Command
%===============================================================================================

% Hervorhebung eines Spezialbegriffs ohne Verweis (f�r �berschriften und den Glossareintrag des
% Begriffs).
\newcommand{\SpecialTermBasic}[1]%
	{\textsc{#1}}
	
% Hervorhebung und Link eines Spezialbegriffs
\newcommand{\SpecialTerm}[2]%
	{\hyperref[#2]{$\left[\rightarrow\right]$} \SpecialTermBasic{\SpecialTermBasic{#1}}}

%###############################################################################################
%###############################################################################################
%
%		Datei-Ende
%
%###############################################################################################
%###############################################################################################