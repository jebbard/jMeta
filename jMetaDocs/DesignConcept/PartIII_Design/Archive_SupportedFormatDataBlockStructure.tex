%===============================================================================================
%		Data Block Structure of Supported Formats
%===============================================================================================

\section{Data Block Structure of Supported Formats}

\OpenIssue{Move to Part IV}{Move to Part IV}

This chapter defines the necessary structure of the data blocks for each supported data format.

Some of the data formats have a lot of similarities to other data formats, mostly due to their history. This is of course true for different format versions. However, it is also true for extensible formats where especially metadata formats are extensions of these other formats. Similarly, \LibName{} has a notion of extensions, where an extension \emph{inherits} data format structures from its hosting format. In this case, the tables in this chapter have a comment on this inheritance mechanism.

The idea of generic blocks has been presented in \SectionLink{sec:GenericDataBlocks}. In the following tables, most of the time only generic blocks are listed. Generic blocks have an ID segment with the string \texttt{GENERIC} followed by the block format name, all in uppercase. In action, the id segment \texttt{GENERIC_...} is replaced by the actual id of the data block. A data format might define multiple generic blocks.

%-----------------------------------------------------------------------------------------------
%		Typical Block Structures
%-----------------------------------------------------------------------------------------------

\subsection{Typical Block Structures}
\label{sec:TypicalBlockStructures}

The goal of \LibName{} is a well-usable API that presents the data formats in a general block structure, that works quite similar for any format. The basis for this is laid by the block types defined by \LibName{}, see \SectionLink{sec:BlockTypesandTypeNames}. Here, this concept is detailed to show some patterns in block hierarchy structures.



%-----------------------------------------------------------------------------------------------
%		ID3v1 Metadata Formats
%-----------------------------------------------------------------------------------------------

\subsection{ID3v1 Metadata Formats}
\label{sec:ID3v1}

ID3v1 and ID3v1.1 are very simple data formats. The following table shows the complete id hierarchy of both formats. Note that only the additions of ID3v1.1 are shown explicitly:

\begin{longtable}{|p{0.2\textwidth}|p{0.2\textwidth}|p{0.2\textwidth}|p{0.2\textwidth}|p{0.2\textwidth}|p{0.2\textwidth}|p{0.2\textwidth}|}
	\hline
	Level 0 & Level 1 & Level 2 & Level 3 & Level 4 & Block Type & Block Format Name\\
	\endhead
	\hline
 	id3v1 & & & & & Tag & \\
	\hline
 	id3v11 & & & & & Tag & \\
	\hline
 	id3v1 & .header & & & & Header & \\
	\hline
 	id3v1 & .header & .id & & & Field & \\
	\hline
 	id3v1 & .payload & & & & Payload & \\
	\hline
 	id3v1 & .payload & .title & & & Attribute & Field \\
	\hline
 	id3v1 & .payload & .artist & & & Attribute & Field\\
	\hline
 	id3v1 & .payload & .album & & & Attribute & Field\\
	\hline
 	id3v1 & .payload & .year & & & Attribute & Field\\
	\hline
 	id3v1 & .payload & .genre & & & Attribute & Field\\
	\hline
 	id3v1 & .payload & .comment & & & Attribute & Field\\
	\hline
 	\multicolumn{7}{\emph{ID3v1.1 inherits all data blocks from ID3v1}}{*}\\
	\hline
 	id3v11 & .payload & .track & & & Attribute & Field\\
	\hline
	\caption{Data block structure of the ID3v1 formats}
	\label{tab:Datablockstructureoftheid3v1formats}
\end{longtable}

\OpenIssue{ID3v1 block issues}{Should the ID3v1 attributes have a payload child block?}

The subject of the ID3v1 tags is the medium itself. Therefor the medium name is returned by the \texttt{getSubject} method. There are no generic blocks in the ID3v1 formats.

%-----------------------------------------------------------------------------------------------
%		ID3v2 Metadata Formats
%-----------------------------------------------------------------------------------------------

\subsection{ID3v2 Metadata Formats}
\label{sec:ID3v2}

The two supported ID3v2 metadata formats are quite similar. The block structure is basically shown in the following table. Note that just a single concrete frame is given as example, preceeded by the generic block definition for ID3v2 attributes. Furthermore, only the additions and changes of ID3v2.4 are shown explicitly:

\begin{longtable}{|p{0.2\textwidth}|p{0.2\textwidth}|p{0.2\textwidth}|p{0.2\textwidth}|p{0.2\textwidth}|p{0.2\textwidth}|p{0.2\textwidth}|}
	\hline
	Level 0 & Level 1 & Level 2 & Level 3 & Level 4 & Block Type & Block Format Name\\
	\endhead
	\hline
 	id3v23 & & & & & Tag & \\
	\hline
 	id3v24 & & & & & Tag & \\
	\hline
 	id3v23 & .header & & & & Header & \\
	\hline
 	id3v23 & .header & .id & & & Field & \\
	\hline
 	id3v23 & .header & .version & & & Field & \\
	\hline
 	id3v23 & .header & .flags & & & Field & \\
	\hline
 	id3v23 & .header & .size & & & Field & \\
	\hline
 	id3v23 & .extHeader & & & & Header & \\
	\hline
 	id3v23 & .extHeader & .size & & & Field & \\
	\hline
 	id3v23 & .extHeader & .flags & & & Field & \\
	\hline
 	id3v23 & .extHeader & .paddingSize & & & Field & \\
	\hline
 	id3v23 & .extHeader & .CRC & & & Field & \\
	\hline
 	id3v23 & .payload & & & & Payload & \\
	\hline
 	id3v23 & .payload & .GENERIC_FRAME & & & Attribute & Frame \\
	\hline
 	id3v23 & .payload & .GENERIC_FRAME & .header & & Header & \\
	\hline
 	id3v23 & .payload & .GENERIC_FRAME & .header & .frameId & Field & \\
	\hline
 	id3v23 & .payload & .GENERIC_FRAME & .header & .size & Field & \\
	\hline
 	id3v23 & .payload & .GENERIC_FRAME & .header & .flags & Field & \\
	\hline
 	id3v23 & .payload & .GENERIC_FRAME & .payload & & Payload & \\
	\hline
 	id3v23 & .payload & .GENERIC_FRAME & .payload & .decompressedSize & Field & \\
	\hline
 	id3v23 & .payload & .GENERIC_FRAME & .payload & .groupId & Field & \\
	\hline
 	id3v23 & .payload & .GENERIC_FRAME & .payload & .encryptionMethod & Field & \\
	\hline
 	id3v23 & .payload & .GENERIC_FRAME & .payload & .encryptionMethod & Field & \\
	\hline
 	id3v23 & .payload & .GENERIC_FRAME & .payload & .encryptionMethod & Field & \\
	\hline
 	\multicolumn{7}{\emph{Alle specific ID3v2.3 frames, in this example the AENC frame, inherit  all data blocks from the GENERIC_FRAME.}}{*}\\
	\hline
 	id3v23 & .payload & .AENC & .payload & .ownerId & Field & \\
	\hline
 	id3v23 & .payload & .AENC & .payload & .previewStart & Field & \\
	\hline
 	id3v23 & .payload & .AENC & .payload & .previewLength & Field & \\
	\hline
 	id3v23 & .payload & .AENC & .payload & .encryptionInfo & Field & \\
	\hline
 	id3v23 & .padding & & & & Padding & \\
	\hline
 	\multicolumn{7}{\emph{ID3v2.4 inherits all data blocks from ID3v2.3, but redefines some blocks, e.g. the \texttt{extHeader} block.}}{*}\\
	\hline
 	id3v24 & .extHeader & & & & Header & \\
	\hline
 	id3v24 & .extHeader & .size & & & Field & \\
	\hline
 	id3v24 & .extHeader & .flagByteCount & & & Field & \\
	\hline
 	id3v24 & .extHeader & .flags & & & Field & \\
	\hline
 	id3v24 & .extHeader & .updateFlagSize & & & Field & \\
	\hline
 	id3v24 & .extHeader & .crcFlagSize & & & Field & \\
	\hline
 	id3v24 & .extHeader & .restrFlagSize & & & Field & \\
	\hline
 	id3v24 & .extHeader & .CRC & & & Field & \\
	\hline
 	id3v24 & .extHeader & .restrictions & & & Field & \\
	\hline
 	id3v24 & .footer & & & & Footer & \\
	\hline
 	id3v24 & .footer & .id & & & Field & \\
	\hline
 	id3v24 & .footer & .version & & & Field & \\
	\hline
 	id3v24 & .footer & .flags & & & Field & \\
	\hline
 	id3v24 & .footer & .size & & & Field & \\
	\hline
	\caption{Data block structure of the ID3v2 formats}
	\label{tab:Datablockstructureoftheid3v2formats}
\end{longtable}

\OpenIssue{ID3v2 block issues}{None yet}

The subject of the ID3v2 tags is the medium itself. Therefor the medium name is returned by the \texttt{getSubject} method. In the table, the AENC frame is shown as an example frame. It has the same structure as the generic block shown. It just additionally specifies its specific payload fields.

%-----------------------------------------------------------------------------------------------
%		APE Metadata Formats
%-----------------------------------------------------------------------------------------------

\subsection{APE Metadata Formats}
\label{sec:APE}

The two supported APE metadata formats are quite similar. The block structure is basically shown in the following table. Note that only the additions and changes of APEv2 are shown explicitly:

\begin{longtable}{|p{0.2\textwidth}|p{0.2\textwidth}|p{0.2\textwidth}|p{0.2\textwidth}|p{0.2\textwidth}|p{0.2\textwidth}|p{0.2\textwidth}|}
	\hline
	Level 0 & Level 1 & Level 2 & Level 3 & Level 4 & Block Type & Block Format Name\\
	\endhead
	\hline
 	apev1 & & & & & Tag & \\
	\hline
 	apev1 & .footer & & & & Header & \\
	\hline
 	apev1 & .footer & .id & & & Field & \\
	\hline
 	apev1 & .footer & .version & & & Field & \\
	\hline
 	apev1 & .footer & .size & & & Field & \\
	\hline
 	apev1 & .footer & .itemCount & & & Field & \\
	\hline
 	apev1 & .footer & .reserved & & & Field & \\
	\hline
 	apev1 & .payload & & & & Payload & \\
	\hline
 	apev1 & .payload & .GENERIC_ITEM & & & Attribute & Item \\
	\hline
 	apev1 & .payload & .GENERIC_ITEM & .header & & Header & \\
	\hline
 	apev1 & .payload & .GENERIC_ITEM & .header & .size & Field & \\
	\hline
 	apev1 & .payload & .GENERIC_ITEM & .header & .flags & Field & \\
	\hline
 	apev1 & .payload & .GENERIC_ITEM & .header & .key & Field & \\
	\hline
 	apev1 & .payload & .GENERIC_ITEM & .payload & & Payload & \\
	\hline
 	apev1 & .payload & .GENERIC_ITEM & .payload & .value & Field & \\
	\hline
 	\multicolumn{7}{\emph{APEv2 inherits all data blocks from APEv1, additionally defining a header and flags for header and footer.}}{*}\\
	\hline
 	apev2 & & & & & Tag & \\
	\hline
 	apev2 & .footer & .flags & & & Field & \\
	\hline
 	apev2 & .header & & & & Header & \\
	\hline
 	apev2 & .header & .id & & & Field & \\
	\hline
 	apev2 & .header & .version & & & Field & \\
	\hline
 	apev2 & .header & .size & & & Field & \\
	\hline
 	apev2 & .header & .itemCount & & & Field & \\
	\hline
 	apev2 & .header & .reserved & & & Field & \\
	\hline
 	apev2 & .header & .flags & & & Field & \\
	\hline
	\caption{Data block structure of the APE formats}
	\label{tab:DatablockstructureoftheAPEformats}
\end{longtable}

\OpenIssue{APE block issues}{Attribute Payload has a single child field or is itself the overall value?}

The subject of the APE tags is the medium itself. Therefor the medium name is returned by the \texttt{getSubject} method. In the table, there is only a generic block. All concrete APE items are perfectly described by the generic block already.

%-----------------------------------------------------------------------------------------------
%		Vorbis Comment Metadata Format
%-----------------------------------------------------------------------------------------------

\subsection{Vorbis Comment Metadata Format}
\label{sec:VorbisComment}

The block structure of a vorbis comment is basically shown in the following table:

\begin{longtable}{|p{0.2\textwidth}|p{0.2\textwidth}|p{0.2\textwidth}|p{0.2\textwidth}|p{0.2\textwidth}|p{0.2\textwidth}|p{0.2\textwidth}|}
	\hline
	Level 0 & Level 1 & Level 2 & Level 3 & Level 4 & Block Type & Block Format Name\\
	\endhead
	\hline
 	vorbisComment & & & & & Tag & \\
	\hline
 	vorbisComment & .header & & & & Header & \\
	\hline
 	vorbisComment & .header & .vendorLength & & & Field & \\
	\hline
 	vorbisComment & .header & .vendorString & & & Field & \\
	\hline
 	vorbisComment & .header & .userCommentCount & & & Field & \\
	\hline
 	vorbisComment & .payload & & & & Payload & \\
	\hline
 	vorbisComment & .payload & .framingBit & & & Field & \\
	\hline
 	vorbisComment & .payload & .GENERIC_USER_COMMENT & & & Attribute & User Comment \\
	\hline
 	vorbisComment & .payload & .GENERIC_USER_COMMENT & .header & & Header & \\
	\hline
 	vorbisComment & .payload & .GENERIC_USER_COMMENT & .header & .size & Field & \\
	\hline
 	vorbisComment & .payload & .GENERIC_USER_COMMENT & .header & .fieldName & Field & \\
	\hline
 	vorbisComment & .payload & .GENERIC_USER_COMMENT & .header & .size & Field & \\
	\hline
 	vorbisComment & .payload & .GENERIC_USER_COMMENT & .payload & & Payload & \\
	\hline
 	vorbisComment & .payload & .GENERIC_USER_COMMENT & .payload & .value & Field & \\
	\hline
	\caption{Data block structure of the Vorbis Comment format}
	\label{tab:DatablockstructureoftheVorbisCommentformats}
\end{longtable}

\OpenIssue{VorbisComment block issues}{Attribute Payload has a single child field or is itself the overall value?}

The subject of the Vorbis Comment is the medium itself. Therefor the medium name is returned by the \texttt{getSubject} method. In the table, there is only a generic block. All concrete Vorbis Comment user comments are perfectly described by the generic block already.

%-----------------------------------------------------------------------------------------------
%		Lyrics3 Metadata Formats
%-----------------------------------------------------------------------------------------------

\subsection{Lyrics3 Metadata Formats}
\label{sec:Lyrics3}

Lyrics3v1 and Lyrics3v2 are very simple data formats, but with some differences. The following table shows the complete id hierarchy of both formats. Note that only the additions and changes of Lyrics3v2 are shown explicitly:

\begin{longtable}{|p{0.2\textwidth}|p{0.2\textwidth}|p{0.2\textwidth}|p{0.2\textwidth}|p{0.2\textwidth}|p{0.2\textwidth}|p{0.2\textwidth}|}
	\hline
	Level 0 & Level 1 & Level 2 & Level 3 & Level 4 & Block Type & Block Format Name\\
	\endhead
	\hline
 	lyrics3v1 & & & & & Tag & \\
	\hline
 	lyrics3v1 & .header & & & & Header & \\
	\hline
 	lyrics3v1 & .header & .id & & & Field & \\
	\hline
 	lyrics3v1 & .footer & & & & Footer & \\
	\hline
 	lyrics3v1 & .footer & .id & & & Field & \\
	\hline
 	lyrics3v1 & .payload & & & & Payload & \\
	\hline
 	lyrics3v1 & .payload & .GENERIC_FIELD & & & Attribute & Field \\
	\hline
 	lyrics3v1 & .payload & .GENERIC_FIELD & .header & & Header & \\
	\hline
 	lyrics3v1 & .payload & .GENERIC_FIELD & .header & .timestamp & Field & \\
	\hline
 	lyrics3v1 & .payload & .GENERIC_FIELD & .payload & & Payload & \\
	\hline
 	lyrics3v1 & .payload & .GENERIC_FIELD & .payload & .lyrics & Field & \\
	\hline
 	\multicolumn{7}{\emph{Lyrics3v2 inherits the basic data blocks from Lyrics3v1, it redefines the GENERIC_FIELD block.}}{*}\\
	\hline
 	lyrics3v2 & & & & & Tag & \\
	\hline
 	lyrics3v2 & .footer & .size & & & Field & \\
	\hline
 	lyrics3v2 & .payload & .GENERIC_FIELD & & & Attribute & Field \\
	\hline
 	lyrics3v2 & .payload & .GENERIC_FIELD & .header & & Header & \\
	\hline
 	lyrics3v2 & .payload & .GENERIC_FIELD & .header & .id & Field & \\
	\hline
 	lyrics3v2 & .payload & .GENERIC_FIELD & .header & .size & Field & \\
	\hline
 	lyrics3v2 & .payload & .GENERIC_FIELD & .payload & & Payload & \\
	\hline
 	lyrics3v2 & .payload & .GENERIC_FIELD & .payload & .value & Field & \\
	\hline
	\caption{Data block structure of the Lyrics3 formats}
	\label{tab:DatablockstructureoftheLyrics3formats}
\end{longtable}

\OpenIssue{Lyrics3 block issues}{Attribute Payload has a single child field or is itself the overall value?}

The subject of the Lyrics3 tags is the medium itself. Therefor the medium name is returned by the \texttt{getSubject} method.

%-----------------------------------------------------------------------------------------------
%		MP3 Container Format
%-----------------------------------------------------------------------------------------------

\subsection{MP3 Container Format}
\label{sec:MP3ContainerFormat}

The following table shows the id hierarchy of the MP3 format, i.e. the MPEG-1 audio elementary stream:

\begin{longtable}{|p{0.2\textwidth}|p{0.2\textwidth}|p{0.2\textwidth}|p{0.2\textwidth}|p{0.2\textwidth}|p{0.2\textwidth}|p{0.2\textwidth}|}
	\hline
	Level 0 & Level 1 & Level 2 & Level 3 & Level 4 & Level 5 & Block Type & Block Format Name\\
	\endhead
	\hline
 	mp3 & & & & & ContainerPart & \\
	\hline
 	mp3 & .header & & & & Header & \\
	\hline
 	mp3 & .header & .frameSync & & & Field & \\
	\hline
 	mp3 & .header & .ID & & & Field & \\
	\hline
 	mp3 & .header & .layer & & & Field & \\
	\hline
 	mp3 & .header & .noProtection & & & Field & \\
	\hline
 	mp3 & .header & .bitRateIndex & & & Field & \\
	\hline
 	mp3 & .header & .samplingFrequency & & & Field & \\
	\hline
 	mp3 & .header & .padding & & & Field & \\
	\hline
 	mp3 & .header & .private & & & Field & \\
	\hline
 	mp3 & .header & .mode & & & Field & \\
	\hline
 	mp3 & .header & .modeExtension & & & Field & \\
	\hline
 	mp3 & .header & .originalOrCopy & & & Field & \\
	\hline
 	mp3 & .header & .copyRight & & & Field & \\
	\hline
 	mp3 & .header & .emphasis & & & Field & \\
	\hline
 	mp3 & .payload & & & & Payload & \\
	\hline
 	mp3 & .padding & & & & Padding & \\
	\hline
 	mp3 & .CRC & & & & Field & \\
	\hline
	\caption{Data block structure of the MP3 format}
	\label{tab:DatablockstructureoftheMP3formats}
\end{longtable}

\OpenIssue{MP3 block issues}{None yet}

There is no extension concept for data blocks in the MP3 format, therefore there are also no generic blocks.

%-----------------------------------------------------------------------------------------------
%		Ogg Container Format
%-----------------------------------------------------------------------------------------------

\subsection{Ogg Container Format}
\label{sec:OggContainerFormat}

The following table shows the id hierarchy of the Ogg format:

\begin{longtable}{|p{0.2\textwidth}|p{0.2\textwidth}|p{0.2\textwidth}|p{0.2\textwidth}|p{0.2\textwidth}|p{0.2\textwidth}|p{0.2\textwidth}|}
	\hline
	Level 0 & Level 1 & Level 2 & Level 3 & Level 4 & Block Type & Block Format Name\\
	\endhead
	\hline
 	ogg & & & & & ContainerPart & Page\\
	\hline
 	ogg & .header & & & & Header & \\
	\hline
 	ogg & .header & .capturePattern & & & Field & \\
	\hline
 	ogg & .header & .streamStructureVersion & & & Field & \\
	\hline
 	ogg & .header & .headerTypeFlag & & & Field & \\
	\hline
 	ogg & .header & .absoluteGranulePosition & & & Field & \\
	\hline
 	ogg & .header & .streamSerialNumber & & & Field & \\
	\hline
 	ogg & .header & .pageSequenceNumber & & & Field & \\
	\hline
 	ogg & .header & .pageChecksum & & & Field & \\
	\hline
 	ogg & .header & .pageSegmentCount & & & Field & \\
	\hline
 	ogg & .header & .segmentTable & & & Field & \\
	\hline
 	ogg & .payload & & & & Payload & \\
	\hline
 	ogg & .payload & .packet & & & Field & Packet \\
	\hline
 	ogg & .payload & .packet & .segment & & Field & Segment\\
	\hline
	\caption{Data block structure of the Ogg format}
	\label{tab:DatablockstructureoftheOggformats}
\end{longtable}

\OpenIssue{Ogg block issues}{None yet}

There is no extension concept for data blocks in the Ogg format, therefore there are also no generic blocks.

%-----------------------------------------------------------------------------------------------
%		RIFF and AIFF Container Formats
%-----------------------------------------------------------------------------------------------

\subsection{RIFF and AIFF Container Formats}
\label{sec:RIFFandAIFFContainerFormats}

RIFF and AIFF are identical structure-wise and therefore listed together in the following table:

\begin{longtable}{|p{0.2\textwidth}|p{0.2\textwidth}|p{0.2\textwidth}|p{0.2\textwidth}|p{0.2\textwidth}|p{0.2\textwidth}|p{0.2\textwidth}|}
	\hline
	Level 0 & Level 1 & Level 2 & Level 3 & Level 4 & Block Type & Block Format Name\\
	\endhead
	\hline
 	riff & & & & & Container & Chunk\\
	\hline
 	riff & .header & & & & Header & \\
	\hline
 	riff & .header & .id & & & Field & \\
	\hline
 	riff & .header & .size & & & Field & \\
	\hline
 	riff & .header & .formType & & & Field & \\
	\hline
 	riff & .payload & & & & Payload & \\
	\hline
 	riff & .payload & .GENERIC_CHUNK & & & ContainerPart & Chunk\\
	\hline
 	riff & .payload & .GENERIC_CHUNK & .header & & Header & \\
	\hline
 	riff & .payload & .GENERIC_CHUNK & .header & .id & Field & \\
	\hline
 	riff & .payload & .GENERIC_CHUNK & .header & .size & Field & \\
	\hline
 	riff & .payload & .GENERIC_CHUNK & .payload & & Payload & \\
	\hline
 	\multicolumn{7}{\emph{AIFF inherits the generic block structure from RIFF, thereby defining a lot of specific blocks that RIFF does not define. These are omitted here for brevity.}}{*}\\
	\hline
 	aiff & & & & & Container & Chunk\\
	\hline
	\caption{Data block structure of the RIFF and AIFF formats}
	\label{tab:DatablockstructureoftheRIFFformats}
\end{longtable}

\OpenIssue{RIFF block issues}{None yet}

%-----------------------------------------------------------------------------------------------
%		Matroska Container Format
%-----------------------------------------------------------------------------------------------

\subsection{Matroska Container Format and Tags}
\label{sec:MatroskaContainerFormats}

The following table shows the id hierarchy of the Matroska format. Note that Matroska is based on the EBML format which therefore defines the generic block for Matroska. The following table only shows a selection of defined top-level blocks:

\begin{longtable}{|p{0.2\textwidth}|p{0.2\textwidth}|p{0.2\textwidth}|p{0.2\textwidth}|p{0.2\textwidth}|p{0.2\textwidth}|p{0.2\textwidth}|}
	\hline
	Level 0 & Level 1 & Level 2 & Level 3 & Level 4 & Block Type & Block Format Name\\
	\endhead
	\hline
 	ebml & & & & & Container & \\
	\hline
 	ebml & .header & & & & Header & \\
	\hline
 	ebml & .header & .elementId & & & Field & \\
	\hline
 	ebml & .header & .size & & & Field & \\
	\hline
 	ebml & .header & .version & & & Field & \\
	\hline
 	ebml & .header & .readVersion & & & Field & \\
	\hline
 	ebml & .header & .maxIdLength & & & Field & \\
	\hline
 	ebml & .header & .maxSizeLength & & & Field & \\
	\hline
 	ebml & .header & .docType & & & Field & \\
	\hline
 	ebml & .header & .docTypeVersion & & & Field & \\
	\hline
 	ebml & .header & .docTypeReadVersion & & & Field & \\
	\hline
 	ebml & .payload & & & & Payload & \\
	\hline
 	ebml & .payload & .GENERIC_ELEMENT & .header & .elementId & Field & \\
	\hline
 	ebml & .payload & .GENERIC_ELEMENT & .header & .size & Field & \\
	\hline
 	\multicolumn{7}{\emph{Matroska uses the EBML format as its basis and therefore inherits the generic block structure from EBML, thereby defining a lot of additional Matroska blocks. All of the payload ids here are inheriting their structure from the GENERIC_ELEMENTs.}}{*}\\
	\hline
 	matroska & .payload & .segment &  &  & Container Part & Segment\\
	\hline
 	matroska & .payload & .metaSeekInformation &  &  & Container Part & \\
	\hline
 	matroska & .payload & .segmentInformation &  &  & Container Part & \\
	\hline
 	matroska & .payload & .track &  &  & Container Part & \\
	\hline
 	matroska & .payload & .chapters &  &  & Container Part & \\
	\hline
 	matroska & .payload & .clusters &  &  & Container Part & \\
	\hline
 	matroska & .payload & .cueingData &  &  & Container Part & \\
	\hline
 	matroska & .payload & .attachment &  &  & Container Part & \\
	\hline
 	matroska & .payload & .tags &  &  & Container Part & \\
	\hline
	\caption{Data block structure of the Matroska format}
	\label{tab:DatablockstructureoftheMatroskaformats}
\end{longtable}

\OpenIssue{Matroska block issues}{None yet}

Note that matroska metadata allows to store multiple so called \emph{Tags} that may refer to the whole or single parts of the matroska stream, or even to the outside. In fact, each of these matroska tags corresponds to a \TERMtag{} for \LibName{} with a unique subject. The subject of the tag is taken from the \texttt{targets} block present in each tag.

%-----------------------------------------------------------------------------------------------
%		QuickTime Container Format
%-----------------------------------------------------------------------------------------------

\subsection{QuickTime Container Format}
\label{sec:QuickTimeContainerFormats}

The following table shows the id hierarchy of the QuickTime format:

\begin{longtable}{|p{0.2\textwidth}|p{0.2\textwidth}|p{0.2\textwidth}|p{0.2\textwidth}|p{0.2\textwidth}|p{0.2\textwidth}|p{0.2\textwidth}|}
	\hline
	Level 0 & Level 1 & Level 2 & Level 3 & Level 4 & Block Type & Block Format Name\\
	\endhead
	\hline
 	\multicolumn{7}{\emph{The QuickTime top-level container parts are themselves GENERIC_ATOMs. Nervertheless they have been fully listed here.}}{*}\\
	\hline
 	qt & & & & & ContainerPart & Atom\\
	\hline
 	qt & .header & & & & Header & \\
	\hline
 	qt & .header & .atomSize & & & Field & \\
	\hline
 	qt & .header & .atomType & & & Field & \\
	\hline
 	qt & .header & .extendedSize & & & Field & \\
	\hline
 	qt & .payload & & & & Payload & \\
	\hline
 	qt & .payload & .GENERIC_ATOM & & & ContainerPart & Atom\\
	\hline
 	qt & .payload & .GENERIC_ATOM & .header & .atomSize & Field & \\
	\hline
 	qt & .payload & .GENERIC_ATOM & .header & .atomType & Field & \\
	\hline
 	qt & .payload & .GENERIC_ATOM & .header & .extendedSize & Field & \\
	\hline
 	qt & .payload & .GENERIC_ATOM & .payload & & Payload & \\
	\hline
 	\multicolumn{7}{\emph{The GENERIC_QT_ATOM inherits all data blocks from the GENERIC_ATOM, therefore it is a more specific GENERIC_ATOM.}}{*}\\
	\hline
 	qt & .payload & .GENERIC_QT_ATOM & & & ContainerPart & QT Atom\\
	\hline
 	qt & .payload & .GENERIC_QT_ATOM & .header & .atomId & Field & \\
	\hline
 	qt & .payload & .GENERIC_QT_ATOM & .header & .childCount & Field & \\
	\hline
 	qt & .payload & .GENERIC_QT_ATOM & .header & .reserved1 & Field & \\
	\hline
 	qt & .payload & .GENERIC_QT_ATOM & .header & .reserved2 & Field & \\
	\hline
 	qt & .payload & .GENERIC_QT_ATOM_CONTAINER & & & ContainerPart & QT Atom Container\\
	\hline
 	qt & .payload & .GENERIC_QT_ATOM_CONTAINER & .header & .reserved & Field & \\
	\hline
 	qt & .payload & .GENERIC_QT_ATOM_CONTAINER & .header & .lockCount & Field & \\
	\hline
 	qt & .payload & .GENERIC_QT_ATOM_CONTAINER & .payload & & Payload & \\
	\hline
	\caption{Data block structure of the QuickTime format}
	\label{tab:DatablockstructureoftheQuickTimeformats}
\end{longtable}

\OpenIssue{QuickTime block issues}{None yet}

For QuickTime, it is useful to define multiple generic blocks matching the usual atoms, but also QT atoms and QT atom containers.

%-----------------------------------------------------------------------------------------------
%		TIFF Container Format
%-----------------------------------------------------------------------------------------------

\subsection{TIFF Container Format}
\label{sec:TIFFContainerFormats}

The following table shows the id hierarchy of the TIFF format:

\begin{longtable}{|p{0.2\textwidth}|p{0.2\textwidth}|p{0.2\textwidth}|p{0.2\textwidth}|p{0.2\textwidth}|p{0.2\textwidth}|p{0.2\textwidth}|p{0.2\textwidth}|}
	\hline
	Level 0 & Level 1 & Level 2 & Level 3 & Level 4 & Level 4 & Block Type & Block Format Name\\
	\endhead
	\hline
 	tiff & & & & & & Container & \\
	\hline
 	tiff & .header & & & & & Header & \\
	\hline
 	tiff & .header & .byteOrder & & & & Field & \\
	\hline
 	tiff & .header & .id & & & & Field & \\
	\hline
 	tiff & .header & .firstIFDOffset & & & & Field & \\
	\hline
 	tiff & .payload & & & & Payload & \\
	\hline
 	tiff & .payload & .GENERIC_IFD & & & & ContainerPart & IFD\\
	\hline
 	tiff & .payload & .GENERIC_IFD & .header & & & Header & \\
	\hline
 	tiff & .payload & .GENERIC_IFD & .header & .entryCount & & Field & \\
	\hline
 	tiff & .payload & .GENERIC_IFD & .payload & & & Payload & \\
	\hline
 	tiff & .payload & .GENERIC_IFD & .footer & .nextIFDOffset & & Field & \\
	\hline
 	tiff & .payload & .GENERIC_IFD & .payload & .GENERIC_IFD_ENTRY & & ContainerPart & IFD Entry\\
	\hline
 	tiff & .payload & .GENERIC_IFD & .payload & .GENERIC_IFD_ENTRY & .tag & Field & \\
	\hline
 	tiff & .payload & .GENERIC_IFD & .payload & .GENERIC_IFD_ENTRY & .fieldType & Field & \\
	\hline
 	tiff & .payload & .GENERIC_IFD & .payload & .GENERIC_IFD_ENTRY & .valueCount & Field & \\
	\hline
 	tiff & .payload & .GENERIC_IFD & .payload & .GENERIC_IFD_ENTRY & .valueOffset & Field & \\
	\hline
 	tiff & .payload & .imagePart & & & Field & \\
	\hline
	\caption{Data block structure of the TIFF format}
	\label{tab:DatablockstructureoftheTIFFformats}
\end{longtable}

\OpenIssue{TIFF block issues}{None yet.}

TIFF is an unusual container format as it defines a pointer structure. Between the pointered IFDs, image data might reside.

%\begin{figure}[H]
%	\centering
%	\includegraphics[width=1.00\textwidth]{Figures/Part_IV/IV_2_InterfaceDiagramDPManagement.pdf}
%	\caption{Interface Diagram of the component \COMPdataPartManagement{}}
%	\label{fig:IV_2_InterfaceDiagramDPManagement}
%\end{figure}
%}

\OpenIssue{Finish diagram}{Finish diagram}

%-----------------------------------------------------------------------------------------------
%		Generic XML Format
%-----------------------------------------------------------------------------------------------

\subsection{Generic XML and HTML Formats}
\label{sec:GenericXMLFormat}

XML can be compared to other container formats as it is a hierarchic format. It is therefore basically simple to define an id hierarchy and this way a hierarchy of data blocks. In this sense, an XML tag is a data block with header, payload and footer. This can be modelled by an id hierarchy as shown in the following table. Note that HTML is very similar to generic XML and therefore also modelled like this:

\begin{longtable}{|p{0.2\textwidth}|p{0.2\textwidth}|p{0.2\textwidth}|p{0.2\textwidth}|p{0.2\textwidth}|p{0.2\textwidth}|p{0.2\textwidth}|p{0.2\textwidth}|}
	\hline
	Level 0 & Level 1 & Level 2 & Level 3 & Level 4 & Level 5 & Block Type & Block Format Name\\
	\endhead
	\hline
 	genericXML & & & & & Container & & \\
	\hline
 	genericXML & .header & & & & Header & & \\
	\hline
 	genericXML & .header & .version & & & & Field & \\
	\hline
 	genericXML & .header & .encoding & & & & Field & \\
	\hline
 	genericXML & .payload & & & & & Payload & \\
	\hline
 	genericXML & .payload & .GENERIC & & & & ContainerPart & Element\\
	\hline
 	genericXML & .payload & .GENERIC & .header & & & & Start Tag\\
	\hline
 	genericXML & .payload & .GENERIC & .header & .id & & & \\
	\hline
 	genericXML & .payload & .GENERIC & .header & .attribute & & & Attribute\\
	\hline
 	genericXML & .payload & .GENERIC & .header & .attribute & .name & & \\
	\hline
 	genericXML & .payload & .GENERIC & .header & .attribute & .value & & \\
	\hline
 	genericXML & .payload & .GENERIC & .payload & & & Payload & \\
	\hline
 	genericXML & .payload & .GENERIC & .footer & & & Footer & End Tag\\
	\hline
 	genericXML & .payload & .GENERIC & .footer & .id & & & \\
	\hline
 	genericHTML & .html & .head & .meta & & & Attribute & \\
	\hline
 	genericHTML & .html & .head & .meta & .name & & Field & \\
	\hline
 	genericHTML & .html & .head & .meta & .content & & Field & \\
	\hline
	\caption{Data block structure of the XML format}
	\label{tab:DatablockstructureoftheXMLformats}
\end{longtable}

\OpenIssue{XML block issues}{None yet}

%-----------------------------------------------------------------------------------------------
%		RDF/XML Format
%-----------------------------------------------------------------------------------------------

\subsection{RDF/XML Format}
\label{sec:RDFXMLFormat}

RDF/XML is a special XML-based metadata format. The following table shows a possible block structure:

\begin{longtable}{|p{0.2\textwidth}|p{0.2\textwidth}|p{0.2\textwidth}|p{0.2\textwidth}|p{0.2\textwidth}|p{0.2\textwidth}|p{0.2\textwidth}|p{0.2\textwidth}|}
	\hline
	Level 0 & Level 1 & Level 2 & Level 3 & Level 4 & Level 5 & Block Type & Block Format Name\\
	\endhead
	\hline
 	rdf & & & & & Container & & \\
	\hline
 	rdf & .header & & & & Header & & \\
	\hline
 	rdf & .payload & & & & & Payload & \\
	\hline
 	rdf & .payload & .description & & & & ContainerPart & \\
	\hline
 	rdf & .payload & .description & .type & & & ContainerPart & \\
	\hline
 	rdf & .footer & & & & Footer & & \\
	\hline
	\caption{Data block structure of the XML format}
	\label{tab:DatablockstructureoftheXMLformats}
\end{longtable}

\OpenIssue{RDF block issues}{None yet}

RDF/XML is basically generic XML. It additionally models so called descriptions as XML tags in the payload. Each description can refer to a type. All other RDF/XML specific things such as about and ID attributes and so one are generically accessible through the generic XML blocks which the rdf block inherits.

%###############################################################################################
%###############################################################################################
%
%		File end
%
%###############################################################################################
%###############################################################################################