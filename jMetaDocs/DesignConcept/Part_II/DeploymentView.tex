%===============================================================================================
%		Deployment View
%===============================================================================================

\chapter{Deployment-Sicht}
\label{sec:DeploymentView}

\OpenIssue{Deployment-Sicht}{Deployment-Sicht}

%
%%-----------------------------------------------------------------------------------------------
%%		General Decisions
%%-----------------------------------------------------------------------------------------------
%
%\section{Mapping Components to Projects and Packages}
%\label{sec:GeneralDecisions}
%
%\OpenIssue{Mapping Components to Projects and Packages}{Mapping Components to Projects and Packages}
%
%%-----------------------------------------------------------------------------------------------
%%		Project Interdependencies
%%-----------------------------------------------------------------------------------------------
%
%\section{Project Interdependencies}
%\label{sec:ProjectInterdependencies}
%
%The following figure shows the dependencies between the \LibName{} development projects:
%
%\begin{figure}[H]
	%\centering
	%\includegraphics[width=1.00\textwidth]{Figures/Part_II/II_4_ProjectInterdependencies.pdf}
	%\caption{\LibName{} development project interdependencies}
	%\label{fig:II_4_ProjectInterdependencies}
%\end{figure}
%
%%-----------------------------------------------------------------------------------------------
%%		Deployment Requirements
%%-----------------------------------------------------------------------------------------------
%
%\section{Deployment Requirements}
%\label{sec:DeploymentRequirements}
%
%\newcommand{\REQUdeployInformationHiding}{REQU\_DEPLOY\_INFORMATION\_HIDING}
%\newcommand{\REQUdeployInternalConfiguration}{REQU\_DEPLOY\_INTERNAL\_CONFIG}
%\newcommand{\REQUdeployExtensionHandling}{REQU\_DEPLOY\_EXTENSION\_HANDLING}
%\newcommand{\REQUdeployProtectedPackages}{REQU\_DEPLOY\_PROTECTED\_PACKAGES}
%\newcommand{\REQUdeployVersions}{REQU\_DEPLOY\_VERSIONS}
%
%The deployment requirements are based on the following constraints:
%\begin{itemize}
	%\item The different actors such as \ACTORuser{} and \ACTORextender{} are allowed to see and use different elements of \LibName{}. This applies to code as well as configuration.
	%\item Due to the extensibility of \LibName{}, extensions must be a separate thing.
%\end{itemize}
%
%The following concrete requirements result from that and the technical \LibName{} implementation:
%\begin{itemize}
	%\item \REQUdeployInformationHiding{}: In general, the information hiding principle here is still effective. The actors must not be able to see, use or instantiate internal implementation classes. This leads to less dependency and the ability to easily exchange implementation code.
	%\item \REQUdeployInternalConfiguration{}: Some vital \LibName{} configuration needs to be made internal so that neither \ACTORuser{}s nor \ACTORextender{}s could change them. This currently applies to the internal configuration and to the component configuration.
	%\item \REQUdeployExtensionHandling{}: Extensions must be visible to \ACTORextender{}s, as they need to be able to add them to an already deployed \LibName{} installation. This implies that extension data is also visible to \ACTORuser{}s. The following artifacts are affected:
	%\begin{itemize}
		%\item The extension point configuration file where all extensions are registered
		%\item The extensions interface (if any)
		%\item The extensions implementation
	%\end{itemize}
	%\item \REQUdeployProtectedPackages{}: All deployed packages must be protected against adding additional code from the outside as well as from decompiling.
	%\item \REQUdeployVersions{}: Every deployment unit must be given a version number that allows to identify its original release build.
%\end{itemize}
%
%%-----------------------------------------------------------------------------------------------
%%		Current State (after stage 1)
%%-----------------------------------------------------------------------------------------------
%
%\section{Current State (after stage 1)}
%\label{sec:CurrentStateAfterstage1}
%
%A end-to-end test has been implemented as automated Ant script. It has been found that:
%\begin{itemize}
	%\item File loading approach was insufficient, as often the files reside in JAR files and have to be loaded as InputStream then rather than as file. Therefore the utility components \CLASSAbstractJAXBLoader{}, \ComponentRegistry{} as well as \COMPextensionManagement{} and \COMPconfiguration{} needed to be adapted.
	%\item JAR files can relate to each other via arbitrary relative paths using class-path: in the MANIFEST file. This behaves like a kind of private class path, i.e. the user of the JAR cannot use or instantiate classes from the other JARs referred this way.
	%\item \OpenIssue{JAXB not on classpath}{It is a mystery currently why JAXB jars need not be on the class path, although they are extensively used by the \ComponentRegistry{} and other components (TODO deploy001).}
	%\item \OpenIssue{jMetaCore on the classpath}{Although jMetaCore which contains all implementation classes is loaded using a URLClassLoader, it also needs to be on the interface JAR's class path (in the MANIFEST). If it is not, the following error occurs: javax.xml.bind.JAXBException: de.je.jmeta.extmanager.impl.jaxb.extpoints doesnt contain ObjectFactory.class or jaxb.index. It seems like JAXB loads the ObjectFactory via reflection, which does not work if the specified package is not on the class path. (TODO deploy002)}
	%\item To ensure nothing at all unnecessary or unexpected is on the class path, a new project 	EasyTag_Integration_Tests has been created, whereas the deployment project EasyTag_Deployment is another one without any class path. EasyTag_Integration_Tests only depends on the JARS deployed into EasyTag_Deployment and runs an easy jUnit test case. It does intentionally not depend directly on any other development projects. It directly refers to the jUnit JAR as well as the LogChecker jar.
%\end{itemize}

%###############################################################################################
%###############################################################################################
%
%		File end
%
%###############################################################################################
%###############################################################################################