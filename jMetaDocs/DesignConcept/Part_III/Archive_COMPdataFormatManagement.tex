%===============================================================================================
%		\COMPdataFormatManagement{} Design
%===============================================================================================

\section{\COMPdataFormatManagement{} Design}
\label{sec:COMPdataFormatManagementImplementationDesign}

This chapter describes the most important features of the \COMPdataFormatManagement{} component in a brief way. Generally, all details and most recent state can be found in the javadoc of the component's classes.

%-----------------------------------------------------------------------------------------------
%		Interface Layer Design
%-----------------------------------------------------------------------------------------------

\subsection{Interface Layer Design}
\label{sec:InterfaceDesignCOMPdataFormatManagement}

The following figure shows the static class diagram of the component \COMPdataFormatManagement{}. The dynamics are shown in \SectionLink{sec:ImplementationDesignCOMPdataFormatManagement}:

\begin{figure}[H]
	\centering
	\includegraphics[width=1.00\textwidth]{Figures/Part_V/V_2_InterfaceCOMPdataFormatManagement.pdf}
	\caption{Interface class diagram of the component \COMPdataFormatManagement{}}
	\label{fig:V_2_InterfaceCOMPdataFormatManagement}
\end{figure}

%-----------------------------------------------------------------------------------------------

\subsubsection{Generic Blocks for Popular Data Formats}
\label{sec:GenericBlocksForPopularDataFormats}

The design is oriented on the domain model.

\begin{longtable}{|p{0.4\textwidth}|p{0.6\textwidth}|}
	\hline
	Data Format & Generic Block\\
	\endhead
	\hline
	ID3v1 & \-- \\
	\hline
	ID3v1.1 & \-- \\
	\hline
	ID3v2.2 & Generic ID3v2.2 frame \\
	\hline
	ID3v2.3 & Generic ID3v2.3 frame \\
	\hline
	ID3v2.4 & Generic ID3v2.4 frame \\
	\hline
	APEv1 & Generic item \\
	\hline
	APEv2 & Generic item \\
	\hline
	VorbisComment & Generic user comment\\
	\hline
	FLAC Metadata & Generic FLAC metadata block \\
	\hline
	RIFF & Generic chunk\\
	\hline
	AIFF & Generic chunk\\
	\hline
	Lyrics3v1 & Generic lyric\\
	\hline
	Lyrics3v2 & Generic field\\
	\hline
	XML & Generic XML tag\\
	\hline
	HTML & Generic HTML tag\\
	\hline
	MP3 & Generic MP3 frame\\
	\hline
	Ogg & Generic Page \\
	\hline
	Matroska & Generic EBML element\\
	\hline
	TIFF & Generic IFD\\
	\hline
	QuickTime & Generic atom, generic atom container\\
	\hline
	\caption{Generic Data Blocks in various data formats}
	\label{tab:GExceptionsdefinedbythiscomponent}
\end{longtable}

%-----------------------------------------------------------------------------------------------

\subsubsection{Field Functions}
\label{sec:FieldFunctions}

Fields often contain information relevant for parsing, i.e. the size of a follow-up chunk of data, its character encoding or byte order. This implies that such fields must be read first to be able to read a specific part of the medium. Usually, such fields are stored before\footnote{In a sense of a smaller absolute medium byte offset.} the data they describe.

\LibName{} uses the concept of \emph{field functions} to deal with these situations. A field function is even more general. It represents a specific relationship between a specific field and a set of other data blocks (referred to by their data block ids). Each field function requires the field to have a specific field type and specific type of interpreted value.
The following field functions are currently defined:

\begin{longtable}{|p{0.25\textwidth}|p{0.25\textwidth}|p{0.25\textwidth}|p{0.25\textwidth}|}
	\hline
	Field Function & Description & Field Type & Interpr. Value Type\\
	\endhead
	\hline
	SIZE\_OF & The field stores the size of exactly one other data block, in bytes. & Numeric & \texttt{java.lang.Long} \\
	\hline
	COUNT\_OF & The field stores the number of occurrences of exactly one other data block, in bytes. & Numeric & \texttt{java.lang.Long} \\
	\hline
	BYTE\_ORDER\_OF & The field stores the byte order of several upcoming data blocks and all of their children. & Enumerated & \texttt{java.nio.ByteOrder} \\
	\hline
	CHARSET\_ORDER\_OF & The field stores the character encoding of several upcoming data blocks and all of their children. & Enumerated & \texttt{java.nio.Charset} \\
	\hline
	ID\_OF & The field stores the actual ID of a generic container. & String & \texttt{java.lang.String} \\
	\hline
	PRESENCE\_OF & A specific flag of the field determines the presence of exactly one other data block. & Flags & \texttt{Flags} \\
	\hline
	TRANSFORMATION\_OF & A specific flag of the field determines the transformation of other data blocks. & Flags & \texttt{Flags} \\
	\hline
	CRC\_32\_OF & The field stores the CRC-32 of a series of other fields. & Binary & \CLASSbinaryValue{} \\
	\hline
\end{longtable}

%-----------------------------------------------------------------------------------------------

\subsubsection{Data Format Specification Rules}
\label{sec:DataFormatSpecificationRules}

- Enumerated and flags fields must have a static field size. Correspondingly, there must not be termination bytes or characters specified for them.
- String, binary, numeric or ANY fields must either have static size, be terminated by a character or bytes, or be referred to by another SIZE\_OF field. Exactly one of these possibilities must be true.
- The static, minimum and maximum size of a numeric field must be smaller than 9 bytes
- The static, minimum and maximum size of a string field must be smaller than Integer.MAX\_VALUE
- The static size of a flags or enumerated field must be smaller than Integer.MAX\_VALUE
- The static, minimum and maximum size of a binary or ANY field must be smaller than Long.MAX\_VALUE
- The static length of a flags field must match its flag specification's byte length
- Any field whose maximum occurrence count is 1 and whose minimum occurrence count is 0 must be referred to by a single PRESENCE\_OF field
- Any field whose maximum occurrence count is bigger than 1 and whose minimum occurrence count is not identical to its maximum occurrence count must be referred to by a single COUNT\_OF field
- The binary and interpreted representations in the enumeration table of an enumerated field must all be unique
- The binary representations in the enumeration table of an enumerated field must all have the static size of the enumerated field
[OPTIONAL - The field type of a field with field function ID\_OF must be string]
- The field type of a field with field function SIZE\_OF must be numeric
- The field type of a field with field function COUNT\_OF must be numeric
- The field type of a field with field function CHARACTER\_ENCODING\_OF must be enumerated and map to Charsets only
- The field type of a field with field function BYTE\_ORDER\_OF must be enumerated and map to ByteOrders only
- The field type of a field with field function PRESENCE\_OF must be flags, the specified flag name must be specified in the flag specification
- The field type of a field with field function TRANSFORMATION\_OF must be flags, the specified flag name must be specified in the flag specification
- The field type of a field with field function CRC\_32\_OF must be binary

%-----------------------------------------------------------------------------------------------
%		Export Layer Design
%-----------------------------------------------------------------------------------------------

\subsection{Export Layer Design}
\label{sec:ExportDesignCOMPdataFormatManagement}

There is no export layer for the \COMPdataFormatManagement{} component.

%-----------------------------------------------------------------------------------------------
%		Implementation Layer Design
%-----------------------------------------------------------------------------------------------

\subsection{Implementation Layer Design}
\label{sec:ImplementationDesignCOMPdataFormatManagement}

The following figure shows the static class diagram of the component \COMPdataFormatManagement{}. The dynamics are shown in \SectionLink{sec:ImplementationDesignCOMPdataFormatManagement}:

\begin{figure}[H]
	\centering
	\includegraphics[width=1.00\textwidth]{Figures/Part_V/V_2_ImplementationCOMPdataFormatManagement.pdf}
	\caption{Implementation class diagram of the component \COMPdataFormatManagement{}}
	\label{fig:V_2_ImplementationCOMPdataFormatManagement}
\end{figure}



%###############################################################################################
%###############################################################################################
%
%		File end
%
%###############################################################################################
%###############################################################################################