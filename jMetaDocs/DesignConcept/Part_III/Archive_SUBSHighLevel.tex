%===============================================================================================
%		\SUBSHighLevel{} Design
%===============================================================================================

\chapter{\SUBSHighLevel{} Design}
\label{sec:SUBSHighLeveldes}

%-----------------------------------------------------------------------------------------------
%		\COMPmetadata{} Design
%-----------------------------------------------------------------------------------------------

\section{\COMPmetadata{} Design}
\label{sec:COMPmetadataImplementationDesign}

This chapter describes the most important features of the \COMPmetadata{} component in a brief way. Generally, all details and most recent state can be found in the javadoc of the component's classes.

%-----------------------------------------------------------------------------------------------
%		Interface Layer Design
%-----------------------------------------------------------------------------------------------

\subsection{Interface Layer Design}
\label{sec:InterfaceDesignCOMPmetadata}

The following figure shows the static class diagram of the component \COMPmetadata{}. The dynamics are shown in \SectionLink{sec:ImplementationDesignCOMPmetadata}:

\begin{figure}[H]
	\centering
	\includegraphics[width=1.00\textwidth]{Figures/Part_V/V_4_InterfaceCOMPmetadata.pdf}
	\caption{Interface class diagram of the component \COMPmetadata{}}
	\label{fig:V_4_InterfaceCOMPmetadata}
\end{figure}

The design is oriented on the domain model.

%-----------------------------------------------------------------------------------------------
%		Export Layer Design
%-----------------------------------------------------------------------------------------------

\subsection{Export Layer Design}
\label{sec:ExportDesignCOMPmetadata}

There is no export layer for the \COMPmetadata{} component.

%-----------------------------------------------------------------------------------------------
%		Implementation Layer Design
%-----------------------------------------------------------------------------------------------

\subsection{Implementation Layer Design}
\label{sec:ImplementationDesignCOMPmetadata}

The following figure shows the static class diagram of the component \COMPmetadata{}. The dynamics are shown in \SectionLink{sec:ImplementationDesignCOMPmetadata}:

\begin{figure}[H]
	\centering
	\includegraphics[width=1.00\textwidth]{Figures/Part_V/V_4_ImplementationCOMPmetadata.pdf}
	\caption{Implementation class diagram of the component \COMPmetadata{}}
	\label{fig:V_4_ImplementationCOMPmetadata}
\end{figure}


%-----------------------------------------------------------------------------------------------

\subsubsection{Steps of \UCreadMetadata{}}
\label{sec:StepsOfUCreadMetadata}

The basic steps of \UCreadMetadata{} are as follows:

\begin{figure}[H]
	\centering
	\includegraphics[width=1.00\textwidth]{Figures/Part_V/V_4_ReadingMetadata.pdf}
	\caption{Schematic flow of the use case \UCreadMetadata{}}
	\label{fig:V_4_ReadingMetadata.pdf}
\end{figure}

The basic flow is very similar, except that \TERMtag{}s might be stored anywhere, therefore location information can be specified as hints for searching, additionally. Similarly, the detailed steps of the use case differ a little bit:

\begin{figure}[H]
	\centering
	\includegraphics[width=1.00\textwidth]{Figures/Part_V/V_4_ReadingMetadataDetailed.pdf}
	\caption{Schematic flow of the use case \UCreadMetadata{} - detailed}
	\label{fig:V_4_ReadingMetadataDetailed.pdf}
\end{figure}

A variation for \TERMtag{}s is that their locations might be specified as offset or file location, e.g. for ID3v1 or ID3v2, or as ``somewhere in a container format'' as e.g. for Matroska \TERMtag{}s, RIFF INFO \TERMtag{}s or the Vorbis Comment.

%-----------------------------------------------------------------------------------------------
%		ID3v2 implementation
%-----------------------------------------------------------------------------------------------

\subsection{ID3v2 implementation}
\label{sec:ID3v2implementation}

%###############################################################################################
%###############################################################################################
%
%		File end
%
%###############################################################################################
%###############################################################################################