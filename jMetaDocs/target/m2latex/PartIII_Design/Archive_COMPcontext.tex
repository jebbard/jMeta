%===============================================================================================
%		\COMPcontext{} Design
%===============================================================================================

\chapter{\COMPcontext{} Design}
\label{sec:COMPcontextDesign}

This chapter describes the most important features of the \COMPcontext{} component in a brief way. Generally, all details and most recent state can be found in the javadoc of the component's classes.

\COMPcontext{} is the user entry point to \LibName{}. It loads required prerequisites and provides access to all other \LibName{} components.

%-----------------------------------------------------------------------------------------------
%		Interface Layer Design
%-----------------------------------------------------------------------------------------------

\section{Interface Layer Design}
\label{sec:InterfaceDesignCOMPcontext}

The following figure shows the static class diagram of the component \COMPcontext{}. The dynamics are shown in \SectionLink{sec:ImplementationDesignCOMPcontext}:

\begin{figure}[H]
	\centering
	\includegraphics[width=1.00\textwidth]{Figures/Part_V/V_4_InterfaceCOMPcontext.pdf}
	\caption{Interface class diagram of the component \COMPcontext{}}
	\label{fig:V_4_InterfaceCOMPcontext}
\end{figure}

\COMPcontext{} consists of a singleton class and a single \ComponentRegistry{} interface.

%-----------------------------------------------------------------------------------------------

\subsection{\COMPcontext{} Requirements}
\label{sec:COMPcontextRequirements}

\newcommand{\REQUcontextStartupTasks}{REQU\_CONTEXT\_STARTUP\_TASKS}
\newcommand{\REQUcontextRobustStartup}{REQU\_CONTEXT\_ROBUST\_STARTUP}
\newcommand{\REQUcontextVerboseLogging}{REQU\_CONTEXT\_VERBOSE\_LOGGING}

The following requirements need to be fulfilled by this component:
\begin{itemize}
	\item \REQUcontextStartupTasks{}: \COMPcontext{} needs to do a rich startup initialization with the following tasks at \LibName{} startup:
	\begin{itemize}
		\item Initialize logging
		\item Log \LibName{} startup information
		\item Load the component configuration loading, and log any errors during this task
		\item Load the user configuration, and log any errors during this task
		\item Load all extensions, and log any errors during this task
		\item Load all components once (initially, as they are singletons), and log any errors 				during this task.
	\end{itemize}
	\item \REQUcontextRobustStartup{}: Startup must be robust to even heavy errors and log all of them in the central log file.
	\item \REQUcontextVerboseLogging{}: Must always log rich information that is vital to error analysis. This is:
	\begin{itemize}
		\item Version, build number and release date of the library runtime at startup
		\item All system properties of the runtime environment
		\item All system properties of the operating system environment
		\item Absence or failed loading of a vital configuration file
		\item Which global configuration values have been loaded and will be used
		\item Which extensions have been loaded and will be used
		\item Which components have been loaded and will be used
	\end{itemize}
\end{itemize}

%-----------------------------------------------------------------------------------------------

\subsection{Meeting \REQUcontextStartupTasks{}, \REQUcontextRobustStartup{} and 
\REQUcontextVerboseLogging{}}
\label{sec:REQUcontextStartupTasks}

\COMPcontext{} just performs all the tasks mentioned in the \REQUcontextStartupTasks{}. Most of the time, there is a catch all statement that catches every possible exception that might occur during these tasks and logs them. If the log file path is invalid or could not be used, logging is done into the console.

The class \CLASSJMetaContext{} does all startup tasks directly and indirectly via its \texttt{getInstance()} method. In detail, the following steps must be done exactly in that order:
\begin{enumerate}
	\item \textbf{Load components:} First, the components are loaded from the component configuration file. This is necessary at first because one part of \COMPcontext{} is also a \ComponentRegistry{} component and therefore is configured in that file. This is also true for all other components, including the essential components \COMPlogging{}, \COMPconfiguriation{} and \COMPextensionManagement{}. Because \COMPlogging{} can first be used after component loading, warnings and problems are not logged yet but merely \emph{recorded} with a special feature of the \COMPlogging{} component. They can be logged later at a appropriate place into the logfile. If any errors occur during this task, \LibName{} startup is aborted with an error message.
	\item \textbf{Instantiate \COMPcontext{}:} This step simply creates a service provider for the single \COMPcontext{} interface. All further tasks are done internally in the code of that service provider's constructor.
	\item \textbf{Log initial startup:} In this step, the basic components \COMPconfiguration{} and \COMPlogging{} are first instantiated. Furthermore, the home directory of \LibName{} is determined. All further paths of configuration files etc. are interpreted relative to this home directory. This is because it is not correct to interpret relative paths as relative to the \emph{current directory}. Because as \LibName{} is a library, the current directory can be just anywhere the JVM has been started from. During and after this step, \COMPlogging{} is not yet initialized. Therefore the same recording of logging must happen during this step. Log records are logged later. In this step, some very important environment information is logged, including release information of \LibName{} itself as well as environmental information such as the \LibName{} home directory, the JVM system properties including class path and information about the operating system. This information can be vital for later error analysis. If any errors occur during this task, \LibName{} startup is aborted with an error message.
	\item \textbf{Load user configuration:} In this step, the user configuration is loaded using \COMPconfiguration{}. Any errors in the configuration are logged. Furthermore, the values of the then loaded parameters are logged. During and after this step, \COMPlogging{} is not yet initialized. Therefore the same recording of logging must happen during this step. Log records are logged later. The reason for this is that the initialization parameters for \COMPlogging{} are configured, therefore configuration must be loaded before initialization of \COMPlogging{}. Errors during this step are simply logged but do not abort \LibName{} startup. In case of an error, simply the default values are taken for any not yet loaded configuration parameter.
	\item \textbf{Setup logging and redirect previous log records:} This step finally initializes the \COMPlogging{} component, using the user configuration parameters loaded in the previous step (log file path and log level). All log records previously recorded with standard loggers are now finally written to the log file in a well-defined order. If any errors occur during this task, \LibName{} startup is aborted with an error message.
	\item \textbf{Load all extensions:} This steps loads all \LibName{} extensions using \COMPextensionManagement{} by instantiating it. This step cannot be done earlier because it requires an already initialized \COMPlogging{}, to ensure that errors are correctly logged. If any errors occur during this task, \LibName{} startup is aborted with an error message.
	\item \textbf{Load other components initially:} This step loads all further \LibName{} components initially by initializing the service providers of their main interface. This step is rather optional, it could be even removed. As soon as the user first uses a component, it gets initialized. This step is nevertheless included as to following reasons: Errors can be detected and logged at startup already and as all components are singletons, their first-time loading will be done at startup already, which does not reduce performance later on anymore. If any errors occur during this task, \LibName{} startup is aborted with an error message.
\end{enumerate}

\REQUcontextVerboseLogging{} is fulfilled by simply enabling forced logging at the beginning of startup and disabling it at the end. Therefore no logged records will be missed to get logged. All the information stated for this requirement is logged in a human-readable way.

%-----------------------------------------------------------------------------------------------
%		Export Layer Design
%-----------------------------------------------------------------------------------------------

\section{Export Layer Design}
\label{sec:ExportDesignCOMPcontext}

There is no export layer for \COMPcontext{}.

%-----------------------------------------------------------------------------------------------
%		Implementation Layer Design
%-----------------------------------------------------------------------------------------------

\section{Implementation Layer Design}
\label{sec:ImplementationDesignCOMPcontext}

The following figure shows the static class diagram of the implementation layer of component \COMPcontext{}:

\begin{figure}[H]
	\centering
	\includegraphics[width=1.00\textwidth]{Figures/Part_V/V_4_ImplementationCOMPcontext.pdf}
	\caption{Implementation class diagram of the component \COMPcontext{}}
	\label{fig:V_4_ImplementationCOMPcontext}
\end{figure}

%-----------------------------------------------------------------------------------------------
%		Test Cases
%-----------------------------------------------------------------------------------------------

\section{Test Cases}
\label{sec:TestCasesCOMPcontext}

The startup is tested. All returned instances must be non-null. There must not be any exceptions during startup. Additionally, there must not be error messages in the log file.

%###############################################################################################
%###############################################################################################
%
%		File end
%
%###############################################################################################
%###############################################################################################